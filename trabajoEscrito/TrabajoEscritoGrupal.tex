\documentclass[11pt]{article}
\usepackage{amssymb,amsmath,amscd}
\usepackage{latexsym,xspace}
%\usepackage[latin1]{inputenc}
\usepackage[utf8]{inputenc}
\usepackage{epsfig}
\usepackage{fancyhdr}
\usepackage{amsfonts}
\usepackage{mathrsfs}
\usepackage{color}
%\usepackage[usenames,dvipsnames,svgnames,table]{xcolor}
\usepackage{xcolor}
%\usepackage{epsfig}
%\usepackage[spanish]{draftcopy}
\usepackage[spanish]{babel}
%\usepackage[ruled,vlined,linesnumbered]{algorithm2e}
\usepackage{algorithm2e}

\addtolength{\hoffset}{-2cm} \addtolength{\textwidth}{4cm}
\addtolength{\voffset}{-1.4cm} \addtolength{\topmargin}{-2cm}
\addtolength{\textheight}{5cm} \addtolength{\leftmargin}{-2cm}

\title{Implementación del Método de Longstaff \& Schwartz para valuación de opciones americanas}
\author{Eduardo Martínez}
\date{Mayo 2022}

\begin{document}

\maketitle

\section{Resumen}

Uno de los objetivos de este proyecto es implementar la técnica de valuación  propuesta por Longstaff, F. \& Schwartz, E. (2001), que se conoce como 
``Aproximación por mínimos cuadrados Monte Carlo'' (LSM por sus siglas en inglés) para estimar una estrategia de ejercicio de opciones Americanas, en el que no hay una expresión cerrada para el precio de la opción. Dicha implementación también incluirá la valuación de un portafolio de dos opciones, por lo tanto la simulación Monte-Carlo involucrará la simulación de trayactorias correlacionadas.


\section{Introducción}

Uno de los objetivos de este proyecto es revisar algunos detalles e implementar la metodología de "aproximación por mínimos cuadrados Monte-Carlo" de Longstaff \& Schwartz (2001) para la valuación de opciones Americanas. El objetivo de ésta 
es describir la evolución del valor de la opción de tal forma que se obtenga la mayor ganancia. Este método es una alternativa al de valuación por simulación 
simple ó el de diferencias finitas.

\section{Definiciones preliminares}
	
\textbf{Definición:}
Un contrato \textbf{forward} es un contrato entre dos partes en el que una  de ellas se compromete a comprar y la otra a vender un activo dado, en una fecha 
futura establecida y a un precio establecido.\\

En los contratos forward, ambas partes están obligadas a ejercer su posición. Si damos flexibilidad a que alguna de las dos partes tenga la posibilidad de no ejercer su posición, surge el concepto de opción.\\
	
\textbf{Definición:}
Una opción \textbf{call (larga)} es un contrato que le da al poseedor de la opción, el derecho más no la obligación de comprar un activo dado, en una fecha 
futura establecida y a un precio establecido.\\
	
Si $K$ es el precio preestablecido (que se conoce como precio \textit{strike}),  $S_t$ es el precio del activo al tiempo $t$ y $T$ es la fecha futura establecida 
en el contrato (que se conoce como fecha de  vencimiento del contrato),  entonces el poseedor de la call ejercerá su derecho a comprar sólo si 
$K < S_T$, i.e. si en el mercado es costoso el activo con respecto al precio strike.\\
	
Haciendo comparación entre los dos contratos anteriores. A diferencia del contrato forward, bajo un contrato call una de las partes nunca pierde (el comprador del contrato) pero la otra tiene una pérdida 
tentativa, es por esto que este tipo de contratos tiene un precio (que se le  conoce como prima de la call). Uno de los objetivos de la teoría de valuacion de opciones es determinar un precio razonable para éstas.\\
	
Al vendededor de un contrato call se le conoce como parte corta de la call y éste tiene la obligación de vender el activo en caso de que la parte larga de 
la call quiere ejercer su derecho a comprarla.\\

Si ahora le damos la posibilidad a la parte vendedora, se tiene un segundo tipo de opción.\\
	
\textbf{Definición:} 
Una opción \textbf{put (larga)} es un contrato que le da al poseedor de la opción, el derecho más no la obligación de vender un activo dado, en una fecha futura  establecida y a un precio establecido.\\
	
Si $K$ es el precio preestablecido (que se conoce como precio strike), $S_t$ es el precio del activo al tiempo $t$ y $T$ es la fecha futura establecida en el contrato (que se conoce como fecha de vencimiento del contrato), entonces el poseedor de la put ejercerá su derecho a vender sólo si $K > S_T$, i.e. si en el mercado es barato el activo con respecto al precio strike.\\
	
De nuevo, comparando el primer contrato y la opcion put; a diferencia del contrato forward, bajo un contrato put una de las partes nunca pierde (el comprador del contrato) pero la otra tiene una pérdida tentativa, es por esto que este tipo de contratos tiene un precio (que se le conoce como prima de la put). Uno de los objetivos de la teoría de valuacion de opciones 
es determinar un precio razonable para éstas.\\
	
Al vendededor de un contrato put se le conoce como parte corta de la put y éste tiene la obligación de comprar el activo en caso de que la parte larga de la put quiere ejercer su derecho a venderla.

\subsection{Estilo de opciones: Europeo, Americano ó Bermuda}

\textbf{Definición:}
\begin{itemize}

\item Se dice que una opción tiene estilo \textbf{Europeo} si sólo se puede ejercer en la fecha 
de vencimiento.
	
\item Se dice que una opción tiene estilo \textbf{Americano} si se puede ejercer en cualquier 
momento previo a la fecha de vencimiento (incluyéndola).
	
\item Se dice que una opción tiene estilo \textbf{Bermuda} si sólo se puede ejercer determinadas fechas previas a la fecha de vencimiento (incluyéndola).
\end{itemize}

\section{Paridad put-call}
	
Sean $C(S,K,T)$ y $P(S,K,T)$ los precios de una call y put Europeas, sobre el mismo activo, con el mismo strike, la misma fecha de vencimiento y $r$ la tasa libre de riesgo. Entonces se  satisface una relación entre dichos precios que se conoce como la \textbf{paridad  put-call}, que se puede plantear en términos matemáticos como:
$$e^{-rT}F_{0,T} + P(S,K,T) = Ke^{-rT} + C(S,K,T),$$ 
donde:

\begin{itemize}

\item $F_{0,T} = e^{rT}S_0$ si el activo no paga dividendos.
\item $F_{0,T} = e^{rT}\left(S_0 - \sum_{j=1}^m Div_j e^{-r t_j}\right)$ si el activo paga dividendos discretos.
\item $F_{0,T} = e^{(r-\delta)T}S_0$ si el activo no paga dividendos continuos.
\end{itemize}

Por ejemplo, para una acción que no paga dividendos, la paridad put-call se escribe como
$$S_0 + P(S,K,T) = Ke^{-rT} + C(S,K,T),$$
que se puede interpretar fácilmente: si se adquiere una put, se está comprando el derecho a vender, por lo tanto se debe contar con el activo ($S_0$) para 
poder venderlo en caso de que así se quiera. Por otro lado, si se adquiere una  call, se está comprando el derecho a comprar, por lo tanto eventualmente se 
podría requerir efectivo $K$ (por tanto se invierten sin riesgo $Ke^{-rT}$).
	
\section{Fórmula de Black \& Scholes para opciones Europeas}
	
Existe un marco de valaución muy popular para la valuación de opciones Europeas: El marco de Black \& Scholes. Bajo este framework se supone que los log-rendimientos del activo subyacente tienen distribución Gaussiana.\\

Para acciones que pagan dividendos continuos, las fórmulas de valuación para opciones Europeas de Black \& Scholes establecen que el precio de una call Europea es 
$$C(S,K,T,\sigma,r,\delta) = S_0e^{-\delta T}\Phi(d_1) - Ke^{-rT}\Phi(d_2),$$

y el precio de una put Europea es
$$P(S,K,T,\sigma,r,\delta) = Ke^{-rT}\Phi(-d_2) - S_0e^{-\delta T}\Phi(d_1)$$
	
con

\begin{itemize}
\item $d_1 = \frac{\log(S_0/K) + (r-\delta+\frac{1}{2}\sigma^2)T}{\sigma\sqrt{T}}$  
\item $d_2 = d_1 - \sigma\sqrt{T}$,
\end{itemize}

donde $r$ es la tasa libre de riesgo con 
composición continua, $\sigma$ es la volatilidad del precio del activo subyacente, $\delta$ es la tasa de dividendos continuos y $\Phi$ es la función de distribución de una variable aleatoria normal estándar.

\section{Valuación de opciones Americanas}

Es difícil estimar el valor en situaciones donde la opción es afectada por más de un factor; es por esto que el método de diferencias finitas y binomial son imprácticos con múltiples factores. En el trading en el día a a día de los muercados suele utilizarse el método de diferencias finitas, sobre-simplificando a un factor, aún cuando se puede demostrar que la opción es afectada por más de un factor.\\

La simulación es una alternativa prometedora ante diferencias finitas ya que permite que las variables del precio del activo subyacente sigan un proceso estocástico general, como ``jump diffusions``; aunque la hipótesis más común es la de movimiento 
Browniano geométrico, que es la que se ocupa también en el framework de B\&S que se comentó anteriormente.\\

Funciona con semi martingalas. La valuacuón a traves de simulaciones tiene la características de que es simple, paralelizable,  transparente y flexible.\\  

Este enfoque es intuitivo al ser determinado por la función de la esperanza condicional del payoff inmediato versus continuar ejerciendo la opción. 
Dicha esperanza condicional se puede estimar con información cross-sectional en la simulación utilizando por ejemplo mínimos cuadrados; ó algún otro algoritmo de aprendizaje estadístico/de máquina.\\

Obteniendo la función de la esperanza condicional para cada fecha del ejercicio se podrá obtener la especificación óptima para ejercer en cada trayectoria.\\
		
Longstaff-Schwartz utiliza esta estrategia de enfocarse directamente en la función 
de esperanza condicional. Otros artículos de investigación también han adoptado 
este enfoque. Sin embargo, Longstaff-Schwartz gusta por ser un enfoque más  pragmático, y más eficiente computacionalmente. Esto es debido a que solo 
realiza regresiones sobre las trayectorias en las que se ejerce la opción.\\

Longstaff-Schwartz proponen un enfoque con buenos resultados de rendimiento y desempeño comparándolo con otras técnicas.

\section{Valuación de opciones Americanas vía la implementación de Longstaff \& Schwartz}
	
Uno de los objetivos es valuar opciones estilo Americano con cash-flows  aleatorios que pueden ocurrir en $[0,T]$. Se supondrá que los payoffs de dichas  opciones está en el espacio de variables aleatorias con varianza finita.\\
	
Se puede demostrar que el valor de una opción Americana se puede representar como una cobertura de Snell; el valor de una opción Americana es igual al valor  máximo del valor presente de los cash-flows de dicha opción, donde el máximo  se toma sobre todos los tiempos de paro con respecto a la filtración, que se 
considera en el marco de valuación.\\
	
\textbf{Notación:}Se denotará por $C(\omega, s; t,T)$ la trayectoria de cash-flows generados por la opción, condicional a que la opción no se ejerce antes (o hasta) del tiempo $t$.\\
	
Uno de los objetivos del algoritmo LS es obtener una aproximación por trayectorias a la regla de paro óptimo que maximiza el valor de la opción 
Americana. Se supondrá que la opción Americana sólo se puede ejercer en $m$ tiempos discretos $0<t_1\leq t_2\leq \ldots \leq t_m=T$ y se considerará la 
política de paro óptimo en cada fecha de ejercicio. Por supuesto, las opciones Americanas pueden ejercerse en cualquier punto en el tiempo, el algortimo LS se puede aproximar el valor de dichas opciones con $m$ suficientemente grande.\\
	
En la fecha de vencimiento, el poseedor de la opción decide ejercerla sólo si ésta tiene un payoff positivo. Sin embargo, en la fecha de ejercicio $t_j$ 
previa a la fecha de vencimiento, el poseedor de la opción debe decidir si ejercer la opción o continuar con ella y replantearse la disyuntiva en la ``siguiente'' fecha de ejercicio.\\
	
El valor de la opción se maximiza por escenarios si el inversionista ejerce  cuando el valor de ejercicio inmediato es mayor ó igual que el valor de continuación.\\
	
Al tiempo $t_j$, el cash-flow de ejercer inmediatamente se conoce (es simplemente el ``payoff'' en ese momento). Sin embargo, el cash-flow de 
continuación no se conoce al tiempo $t_j$. Además, la teoría de valuación de  no-arbitraje considera al valor de continuación como la esperanza del valor 
presente esperado de los cash-flows restantes $C(\omega,s;t:j,T)$,  donde el valor esperado es con respecto a la medida riesgo neutro $\mathbb{Q}$. Es decir, el valor de continuación al tiempo 
$t_j$, $V_c(\omega,t_j)$, se puede escribir como
	
$$V_c(\omega,t_j) = \mathbb{E}_{\mathbb{Q}}\left[\sum_{i=j+1}^m \exp\left\{-\int_{t_j}^{t_i}r(\omega,s)ds\right\}C(\omega,t_i;t_j,T)|\mathcal{F}_{t_j}\right],$$ 

donde $r(\omega,t)$ es la tasa libre de riesgo (posiblemente estocástica, pero en este trabajo se considerará constante) y la esperanza que se considera es condicional a la información  $\mathcal{F}_{t_j}$ al tiempo $t_j$. Con esta representación,  el problema de ejercicio óptimo se reduce a comparar el ejercicio inmediato con esta esperanza condicional; y entonces ejercer cuando el valor de ejercicio sea positivo y mayor ó igual que la esperanza condicional.

\subsection{El algoritmo LS}
	
La metodología LS utiliza mínimos cuadrados para aproximar la función esperanza condicional en $t_m, t_{m-1},\ldots,t_1$. Se trabaja hacia atrás ya que las trayectorias de los cash-flows $E(\omega,s:t,T)$ generadas por la opción se  definen de manera recursiva. Puede ocurrir que $E(\omega,s;t_j,T)$ sea diferente  de $E(\omega,s;t_{j+1},T)$ ya que puede ser óptimo parar al tiempo $j_{j+1}$ y por ende cambiar todos los cash-flows subsecuentes en toda la trayectoria  observada $\omega$. En particular, la tiempo $t_{m-1}$ se supone que la forma funcional desconocida de $V_c(\omega;t_{m-1})$ se puede representar como una combinación lineal de un conjunto numerable de funciones base $\mathcal{F}_{t_{m-1}}$-medibles, i.e. variables aleatorias con respecto a la $\sigma$-álgebra $\mathcal{F}_{t_{m-1}}$.\\
	
Esta suposición se puede justificar de manera formal, con alguna hipótesis. Por ejemplo, si la esperanza condicional es una variable aleatoria con varianza 
finita (con respecto a alguna medida), i.e. está en $L^2$ y como $L^2$ es un espacio de Hilbert, entonces tiene una base ortonormal numerable y dicha esperanza condicional se puede representar como una función lineal de los elementos de dicha base. Una posible elección de las funciones base es el conjunto de polinomios de Laguerre ponderados:

\begin{itemize}
\item $L_0(x) = \exp\{-x/2\}$  
\item $L_1(x)=\exp\{-x/2\}(1-x)$  
\item $L_2(x)=\exp\{-x/2\}(1-2x+\frac{1}{2}x^2)$  
\item $L_n(x)=\exp\{-x/2\}\frac{e^x}{n!}\frac{\partial^2}{\partial x^n}(x^ne^{-x})$
\end{itemize}
	
Con esta especificación $V_c(\omega,t_{m-1})$ se puede expresar como $V_c(\omega,t_{m-1})=\sum_{i=0}^{\infty}\beta_iL_i(S),$ donde $S$ es el valor del activo subyacente de la opción y los coeficientes $\beta_i$'s son constantes. Otros tipos de funciones base son los polinomios de Hermite, Legendre, Chebyshev, Gegenbauer y Jacobi. Algunas pruebas numéricas han probado que series de Fourier y potencias simples también dan resultados adecuados y en este trabajo, por simplicidad se usarán funciones polinomiales simples (como en el artículo original).\\

Para implementar la metodología LSM se aproxima $V_0(\omega,t_{m-1})$ utilizando las primeras $M$ funciones base. Se denotará por $V_{c,M}(\omega;t_{m-1})$ a esta aproximación.
$V_{c,M}(\omega;t_{m-1})$ se estima a partir de hacer regresión de $E(\omega,s;t_{m-1},T)$ con las funciones base como regresores en aquellas trayectorias en las que la opción tiene un payoff positivo al tiempo $t_{m-1}$. Sólo se consideran las trayectorias en las que el payoff es positivo.\\

Al sólo considerar las trayectorias con payoff positivo, se restringe la región en la que se debe estimar la esperanza condicional y se necesitan
menor regresores para obtener una aproximación adecuada de la función esperanza condicional.\\

Como los valores de las funciones base son independientes e idénticamente distribuidas a lo largo de los escenarios, no se requieren suposiciones mayores para la existencia de los momentos y por lo tanto, el valor ajustado de esta regresión,
$\widehat{V}_{c,M}(\omega,t_{m-1})$ converge en media cuadrática y en probabilidad a $V_{c,M}(\omega,t_{m-1})$ conforme el número de
trayectorias con payoff positivo en la simulación tiende a infinito. Además, $\widehat{V}_{c,M}(\omega,t_{m-1})$ es el mejor estimador lineal insesgado de $V_{c,M}(\omega,t_{m-1})$ en términos de métrica
de media cuadrática.\\

Se puede resumir el algoritmo en general de la siguiente forma:\\

\begin{algorithm}[H]
\DontPrintSemicolon
\SetAlgoLined
\SetKwInOut{Input}{Input}\SetKwInOut{Output}{Output}
\Input{Precio strike, $K$; tasa libre de riesgo, $r$; número de pasos $T$; numero de trayectorias, $n$}
\Output{Precio de una opción Americana}
\BlankLine
Generar $n$ trayectorias, cada una de longitud $T$\\
\For{j $\leftarrow$ $T-1$ \textbf{to} 1}{
    \For{cada escenario $\omega$}{
	Encontrar un estimado del valor de continuación en $t_j$ para $\omega$, $V_c(\omega,t_j)$\;
	Calcular el valor de ejercicio en $t_j$ para el escenario $\omega$ es $E(\omega,t_j)$, con $E(\omega,t_j)=\max\{S_{t_j}(\omega)-K,0\}$ para las opciones call y $E(\omega,t_j)=\max\{K-S_{t_j}(\omega),0\}$ para las opciones put, donde $S_{t_j}(\omega)$ es el precio simulado del subyacente al tiempo $t_j$ (en el escenario $\omega$)\;
	\eIf{$V_c(\omega,t_j)\leq E(\omega,t_j)$}{
        La estrategia óptima al tiempo $t_j$ es ejercer\;
    }{
        La estrategia es continuar\;
    }
	}
}
\caption{Algoritmo LS para valuación de opciones Americanas}
\end{algorithm}

\ \\

Para el caso de regresión por mínimos cuadrados se puede utilizar la teória detrás de ésta, así como algunas optimizaciones numéricas para hacer una implementación más efectiva (basícamente la descomposición SVM para la estimación de la matriz inversa para determinar los coeficientes de regresión). Sea $A_j$ la matriz de diseño para el $j$-ésimo tiempo de simulación, donde $A_j(i,k) = L_k(S(t_j,\omega_i))$ para el caso de que los regresores sean bases polinomiales (polinomios de Legendre, Laguerre, Chebyshev, etc.) ó bien $A_j(i,k) = S^k(t_j,\omega_i)$ para el caso de que los regresores sean las primeras potencias de polinomios simples y $\omega_i$ es el $i$-ésmo escenario, $i\in\{1,\ldots,n\}$. Sea $X_j = [\beta_k]_j$ el vector de coeficientes ajustado para las funciones regresoras del $j$-ésimo tiempo de simulación. El objetivo es encontrar un vector de coeficientes $X_j$ que minimice $||A_jX_j - Y_j||^2_2$. El vector con valores $[Y_j]_i$ se obtiene al traer a valor presente el valor de la opción en el escenario $i$ del tiempo $t_{j+1}$ al tiempo $t_j$. Una vez que se determinan los coeficientes de regresión $\hat{\beta}_k$, entonces el $i$-ésimo escenario de los valores de la opción se actualizan de la siguiente manera:\\

Se compara el valor de continuación estimado $\hat{V}_c(\omega_i,t_j)$ con valor de ejercio en $S(t_j,\omega_i)$, i.e. el valor intrínseco de la opción en $(t_j,\omega_i)$. Si el valor de ejercicio es mayor que el valor intrínseco, entonces se usa el valor de continuación estimado como la actualización del precio en el escenario $i$; en otro caso se usa el valor presente al tiempo $t_{j+1}$ (traido a valor presente con la tasa $r$).\\

Sea $U\Sigma V^{\top}$ la descomposición en valores singulares de $A_j$, donde $U\in\mathbb{R}^{n\times n}$, $V\in\mathbb{R}^{m\times m}$ son matrices ortogonales y $\Sigma\in\mathbb{R}^{n\times m}$ es una matriz diagonal. Entonces, $||A_jX_j - Y_j||_2$ se minimiza en $X_j = (A_j^{\top}A_j)^{-1}A_j^{\top}Y_j = (V\Sigma^{-1}U^{\top})Y_j$.\\

\textbf{Observación:} Esta descomposición existe en el caso de polinomios simples ya que $A_j^{\top}A_j$ es una matriz de Vandermonde.\\

Con este mecanismo de regresión se obtiene una estimación directa de los coeficientes $\beta_k$ para la función esperanza condicional $\hat{V}_c(\omega,t_k)$, que se expresa como una combinación lineal ya sea de las funciones polinomiales base ó simplemente de polinomios simples.\\

\begin{algorithm}[H]
\DontPrintSemicolon
\SetAlgoLined
\SetKwInOut{Input}{Input}\SetKwInOut{Output}{Output}
\Input{Valores para todos los escenarios al tiempo $t_j$, $M$ funciones base $L_1,\ldots,L_M$, vector de precios de la opción ($Y_j$)}
\Output{$V_c(\omega,t_j)$}
\BlankLine
Guardar los valores al tiempo $t_j$ para todas las trayectorias, i.e. hacer $S_j(i)=S(t_j,\omega_i)$\;
Eliminar los renglones, $i$, tales que $E(\omega,t_j)$ sea igual a 0, con $E(\omega_i,t_j)=\max\{S_{t_j}(\omega_i)-K,0\}$ para las opciones call y $E(\omega_i,t_j)=\max\{K-S_{t_j}(\omega_i),0\}$ para las opciones put.\;
Mediante mínimos cuadrados, resolver el sistema de ecuaciones $Y_j = \sum_{k=1}^M \beta_kL_k(S_j)$ para determinar $\hat{\beta}_1,\ldots,\hat{\beta}_M$. Esto se hace a partir de la subrutina:\;
Determinar la descomposición SVD de la matriz de diseño $A_j$ definida como $A_j(i,k)=L_k(S(t_j,\omega_i))$, i.e. encontrar matrices $U,V$ tales que $A_j = U\Sigma V^{\top}$;
Calcular $\hat{\beta}_1,\ldots,\hat{\beta}_M$ como $[\hat{\beta}]_j = (V\Sigma^{-1}V^{\top})Y_j$\;
\Return{$\sum_{k=1}^M \hat{\beta}_kL_k(S_j)$}
\caption{Estimación del valor de continuación al tiempo $t_j$ para cada escenario $\omega$ vía bases polinomiales}
\end{algorithm}

\begin{algorithm}[H]
\DontPrintSemicolon
\SetAlgoLined
\SetKwInOut{Input}{Input}\SetKwInOut{Output}{Output}
\Input{Valores para todos los escenarios al tiempo $t_j$, $M$ funciones base $L_1,\ldots,L_M$, vector de precios de la opción ($Y_j$)}
\Output{$V_c(\omega,t_j)$}
\BlankLine
Guardar los valores al tiempo $t_j$ para todas las trayectorias, i.e. hacer $S_j(i)=S(t_j,\omega_i)$\;
Eliminar los renglones, $i$, tales que $E(\omega,t_j)$ sea igual a 0, con $E(\omega_i,t_j)=\max\{S_{t_j}(\omega_i)-K,0\}$ para las opciones call y $E(\omega_i,t_j)=\max\{K-S_{t_j}(\omega_i),0\}$ para las opciones put.\;
\For{g $\leftarrow$ 1 \textbf{to} 10}{
Mediante mínimos cuadrados, resolver el sistema de ecuaciones $Y_j = \sum_{k=1}^g \beta_k S_j^k$ para determinar $\hat{\beta}_1,\ldots,\hat{\beta}_g$. Esto se hace a partir de la subrutina:\;
Determinar la descomposición SVD de la matriz de diseño $A_j$ definida como $A_j(i,k)=S^k(t_j,\omega_i)$, i.e. encontrar matrices $U,V$ tales que $A_j = U\Sigma V^{\top}$;
Calcular $\hat{\beta}_1,\ldots,\hat{\beta}_g$ como $[\hat{\beta}_k]_j = (V\Sigma^{-1}V^{\top})Y_j$\;
Hacer $\hat{Y}_{j,g}=\sum_{k=1}^g \hat{\beta}_k S_j^k$\;
Encontrar $g^*$ tal que $(\hat{Y}_{j,g}-Y_j)^2/(n_j-g-1)$ sea mínimo, i.e. $g$ que mínimice la suma de cuadrados de los residuales para el $g$-ésimo polinomio, donde $n_j$ es el número de renglones distintos donde $Y_j$ es diferente de 0.\;
\Return{$g^*$}}
Mediante mínimos cuadrados, resolver el sistema de ecuaciones $Y_j = \sum_{k=1}^{g^*} \beta_k S_j^k$ para determinar $\hat{\beta}_1,\ldots,\hat{\beta}_{g^*}$. Esto se hace a partir de la subrutina:\;
Determinar la descomposición SVD de la matriz de diseño $A_j$ definida como $A_j(i,k)=S^k(t_j,\omega_i)$, i.e. encontrar matrices $U,V$ tales que $A_j = U\Sigma V^{\top}$;
Calcular $\hat{\beta}_1,\ldots,\hat{\beta}_{g^*}$ como $[\hat{\beta}_k]_j = (V\Sigma^{-1}V^{\top})Y_j$\;
\Return{$\sum_{k=1}^{g^*} \hat{\beta}_kS_j^k$}
\caption{Estimación del valor de continuación al tiempo $t_j$ para cada escenario $\omega$ vía polinomios simples}
\end{algorithm}

\section{Simulación de las trayectorias}

\subsection{Estimación de parámetros para las simulaciones}

Sea $(S_t)_{\geq 0}$ un proceso estocástico que representa el precio de un activo, i.e. $S_t$ es el precio de la acción al tiempo $t$.\\
	
Se dice que el proceso $(S_t)_{\geq 0}$ es un movimiento Browniano geométrico con parámetros $\mu,\sigma>0$ si es la solución de la ecuación diferencial estocástica $$dS_t = \mu S_t dt + \sigma S_t dW_t$$ para algún valor inicial $S_0$.\\
	
Esta expresión se puede reescribir como $$\frac{dS_t}{S_t} = \mu dt + \sigma dW_t$$
	
Se puede dar una interpretación de la ecuación anterior de la siguiente manera 
$$\frac{S_{t+dt}-S_t}{S_t}=\frac{dS_t}{S_t} = \mu dt + \sigma dW_t = 
\mbox{contribución determinista + constribución estocástica}$$
	
donde se supone que la contribución determinista es proporcional y la parte estocástica tiene ley Gaussiana.\\
	
A la constante $\mu$ se le conoce como drift del proceso y $\sigma$ se conoce  como parámetro volatilidad o de difusión.\\
	
Se puede demostrar que una solución explícita para la ecuación diferencial estocástica anterior es 
$$S_t =  S_0 \exp\left\{\alpha t + \sigma W_t\right\} = S_0 \exp\left\{\mu t -\frac{1}{2}\sigma^2t + \sigma W_t\right\},$$ donde $\alpha = \mu-\frac{1}{2}\sigma^2$.
	
\textbf{Observación:}
Si $\sigma = 0$, i.e. no hay ruido estocástico, entonces la ecuación se convierte en 
$$\frac{dS_t}{S_t} = \mu dt,$$ 
equivalentemente $$\frac{d}{dt}\log(S_t) = \mu$$
y por lo tanto $$S_t = S_0 e^{\mu t}.$$
	
Nótese que la diferencia entre la solución determinista y no determinista es el término $\sigma W_t-\frac{1}{2}\sigma^2t$.\\
	
A partir de la solución $$S_t = S_0 \exp\left\{\mu t -\frac{1}{2}\sigma^2t + \sigma W_t\right\}$$ 
se puede ver que $S_t$ es la exponencial de un movimiento Browniano,  i.e. es la exponencial de una variable aleatoria con distribución normal. 
Equivalentemente $S_t$ tiene distribución log-normal. 
Esta es una de las razones por las que el movimiento Browniano es adecuado para aplicaciones financieras.\\
	
Considérese $t_0=0 < t_1 < t_2 <\ldots<t_n=T$ puntos en el horizonte de tiempo $[0,T]$. Defínase $Y_1,\ldots,Y_n$ como
	
$$Y_i = \frac{S_{t_i}-S_{t_{i-1}}}{S_{t_{i-1}}}=\frac{S_{t_i}}{S_{t_{i-1}}}-1$$
	
Si se considera la expansión de Taylor de la función $\log(1+z)$, se tiene que para $z$ suficientemente pequeña
$$\log(1+z) = z-\frac{1}{2}z^2 + o(z) \approx z$$

Entonces si $X_i = 1+ Y_i$, entonces 
$$\log(X_i) = \log(1+Y_i) \approx Y_i = \log(S_{t_i})-\log(S_{t_{i-1}})$$
Esto significa que los rendimientos exactos son casi idénticos a los log-rendimientos
$$X_i = \log(S_{t_i})-\log(S_{t_{i-1}}),\ i\in\{1,\ldots,n\}$$
	
Sea $\Delta t = t_i - t_{i-1}$ para $i\in\{1,\ldots,n\}$. Generalmente $\Delta t = \frac{1}{252}$ si se consideran periodos anuales

$$X_i + \ldots + X_{i+n-1}=\sum_{k=i}^{i+n-1}[\log(S_{t_i})-\log(S_{t_{i-1}})]=\log(S_{i+n-1})-\log(S_{i-1})$$
	
Finalmente
$$X_i = \log(S_{t_i})-\log(S_{t_{i-1}}) = \log\left(\frac{S_{t_i}}{S_{t_{i-1}}}\right)$$
$$=\alpha \Delta t + \sigma[W_{t_i}-W_{t_{i-1}}]\sim N(\alpha t , \sigma^2 t)$$
Es decir,
$$X_i = \log\left(\frac{S_{t_i}}{S_{t_{i-1}}}\right) = \alpha \Delta t + \sigma\sqrt{\Delta t}Z,$$
donde $Z\sim N(0,1)$ además, para $i\neq j$ $X_{t_i}$ es independiente de $X_{t_j}$ pues cada una 
depende de incrementos disjuntos del movimiento.\\

Gracias al análisis anterior se puede considerar a $X_1,\ldots,X_n$  como variables aleatorias independientes, idénticamente distribuidas 
$N(\alpha\Delta t,\sigma^2t \Delta t)$. Los estimadores máximo verosímiles para 
$\alpha$ y $\sigma^2$ son
$$\widehat{\alpha}=\frac{1}{\Delta t}\frac{1}{n}\sum_{i=1}^n X_i$$
$$\widehat{\sigma}^2 = \frac{1}{\Delta t}\frac{1}{n-1}\sum_{i=1}^n(X_i-\bar{X})^2=\frac{\widehat{S}^2}{\Delta t}$$
	
De aquí que un estimador para $\mu$ es
$$\widehat{\mu} = \widehat{\alpha} + \frac{1}{2}\widehat{\sigma}^2$$
	
\textbf{Observación:}
Nótese que 
$$\sum_{i=1}^n X_i = \sum_{i=1}^n [\log(S_{t_i})-\log(S_{t_{i-1}})]=\log(S_{t_n})-\log(S_0)$$
	
De aquí que
$$\widehat{\alpha}=\frac{1}{n\Delta t}[\log(S_{t_n})-\log(S_0)]$$
y entonces
	
$$\widehat{\mu} =\frac{\log(S_{t_n})-\log(S_0)}{n\Delta t}+ \frac{1}{\Delta t}\frac{1}{n-1}\sum_{i=1}^n(X_i-\bar{X})^2$$
$$\widehat{\sigma}^2 = \frac{1}{\Delta t}\frac{1}{n-1}\sum_{i=1}^n(X_i-\bar{X})^2=\frac{\widehat{S}^2}{\Delta t}$$
	
\subsection{Resultados de convergencia}

Mediante el algoritmo LS, se tiene un mecanismo sencillo de aproximar la estrategia de ejercicio óptimo para una opción estilo Americano (y Bermuda también).\\

Hay algunos resultados de convergencia que garantizas que dicha aproximación efectivamente es posible.\\

\textbf{Proposición:}
Para cualesquiera $m,M\in \mathbb{N}_+$ y $\beta\in\mathbb{R}^{m\times(m-1)}$ que representa el vector de coeficientes para las $M$ funciones base en cada uno de las $m-1$ fechas de ejercicio (anticipado), sea $LSM(\omega;M,m)$ el valor presente que se obtiene de la regla LS en caso de que el valor de ejercicio sea positivo y mayor ó igual que $\widehat{V}_{c,M}(\omega_i,t_j)$ (que se define a partir de $\beta$). Entonces, se satisface

$$V(X)\geq \frac{1}{n}\sum_{i=1}^n LSM(\omega_i;M,m),\ \mbox{casi seguramente},$$

donde $V(X)$ es el verdadero valor de la opción Americana.\\

Esta proposición establece que el algoritmo LS implica una regla de paro para opciones Americanas. Sin embargo el valor de dicha opción depende de la regla de paro que maximiza su valor. Otras reglas de paro,
incluyendo la que se obtiene con el algortimo LS, llevan a valores menores o iguales que la que induce la regla de paro óptimo.\\

Este resultado es muy útil pues mediante éste se determina un criterio objetivo para establecer algún tipo de convergencia. Con este criterio se puede obtener algún parámetro o métrica que permita determinar el número de funciones base que se necesitan para obtener una aproximación más exacta: incrementar el valor de $m$ hasta que el precio que se
obtiene con el algoritmo LS no presente cambios significativos. Este es una mejora algorítmica pues otros algoritmos que simplemente llevan a valor presente el valor de continuación a posteriori no permiten este monitoreo de la convergencia.\\

Aunque se quisiera algún resultado de convergencia más fuerte para el algorimo LS, desde su construcción esto es difícil pues se necesita considerar límites con respecto al número de discretizaciones $m$, el
número de funciones base $M$ (los regresores) y el número de trayectorias simuladas ($n\rightarrow \infty$). Además, se debe considerar los efectos de la estimación en las reglas de paro desde el
tiempo $t_{m-1}$ a $t_1$.\\

\textbf{Proposición:} 
Supóngase que el precio de una opción Americana
sólo depende de la variable de precio del activo subyacente $X$ (que puede tomar valores en $(0,\infty)$) cuya dinámica estocástica sigue un proceso de Markov. Además, supóngase que sólo se puede ejercer la opción en los tiempos $t_1$ y $t_2$ y que la función esperanza condicional $V_c(\omega;t_1)$ es absolutamente continua y que las dos
siguientes integrales existen en $\mathbb{R}$
$\int_0^{\infty}e^{-x}V^2_c(\omega;t_1)dx, \int_0^{\infty}e^{-x}V^2_X(\omega;t_1)dx$
Entonces, para cualquier $\epsilon>0$, existe $M\in\mathbb{R}$ tal que

$$\lim_{n\rightarrow \infty} \mathbb{P}\left(\left|V(X)-\frac{1}{n}\sum_{i=1}^n LSM(\omega_i;M,m)\right|>\epsilon\right)=0$$

Este resultado garantiza que si se escoge una valor de $M$ suficientemente grande y se hace $n\rightarrow \infty$, con el algoritmo LSM se obtiene un valor para la opción Americana cercano en menos de una distancia de $\epsilon$ del verdadero valor.

\subsubsection{Estimación de trayectorias correlacionadas}

Si se quiere un derivado sobre un portafolio de dos acciones de la forma

$$P = wS^{[1]} + (1-w)S^{[2]}$$

Se supondrá una dinámica estocástica para cada uno de los activos

$$dS_t^{[1]} = rS_t^{[1]}dt + \sigma_1S_t^{[1]}dZ_t^{[1]}$$
$$dS_t^{[2]} = rS_t^{[2]}dt + \sigma_2S_t^{[2]}dZ_t^{[2]}$$

con $dZ_t^{[1]}dZ_t^{[1]} = \rho dt$, donde $\rho$ es la correlación entre los activos.\\

Es decir,

$$S_h^{[1]} = S_0^{[1]}\exp\left\{(\mu_1 - \frac{1}{2}\sigma_1^2)h + \sigma_1\sqrt{h}\epsilon_1\right\} $$
$$S_h^{[2]} = S_0^{[2]}\exp\left\{(\mu_2 - \frac{1}{2}\sigma_2^2)h + \sigma_2\sqrt{h}\epsilon_2\right\} $$

donde $\epsilon_1 = Z_1$ y $\epsilon_2 =\rho Z_1 + \sqrt{1-\rho^2}Z_2$.\\

Entonces, para simular trayectorias correlacioadas, lo único que cambia es la construcción de las variables Gaussianas $\epsilon_1,\epsilon_2$.\\

Por lo tanto, el portafolio $P = wS^{[1]} + (1-w)S^{[2]}$ se considera como cualquier otros activo, del que podemos obtener trayectorias simuladas y por lo tanto implementar valuación por simulación Monte-Carlo, en particular la que protagoniza este proyecto: la Longstaff \& Schwartz.\\

En este caso, también se satisface que el rendimiento del portafolio es

$$r_P = wr^{[1]} + (1-w)r^{[2]},$$

donde $r^{[1]},r^{[2]}$ son los rendimientos de los activos 1 y 2, respectivamente.\\

Con propiedades elementales de media y varianza se tiene que

$$\mathbb{E}(r_P) = w\mathbb{E}(r^{[1]}) + (1-w)\mathbb{E}(r^{[2]}),$$

$$Var(r_P) = w^2Var(r^{[1]}) + (1-w)^2Var(r^{[2]})+2w(1-w)\rho\sigma_1\sigma_2.$$

\section{Implementación en Python}

Se hizo la implementación funcionalizada del algoritmo de Longstaff-Schwatz tanto para un actvo como para un portafolio de dos activos correlacionados y, para términos comparativos, se implementó además la valuación por árbol binomial también para un activo individual como para un portafolio de dos activos correlacionados.\\

Para esto, se implementaron las siguientes funciones:

\begin{itemize}

\item \begin{verbatim}
calcula_media(ticker, fecha_inicio, fecha_fin)
\end{verbatim}
Se conecta a Yahoo Finance descarga el histórico del ticker y devuelve la media de los log-rendimientos.

\item \begin{verbatim}
calcula_desv_est(ticker, fecha_inicio, fecha_fin)
\end{verbatim}
Se conecta a Yahoo Finance descarga el histórico del ticker y devuelve la varianza de los log-rendimientos.

\item \begin{verbatim}
calcula_correlacion(ticker1, ticker2, fecha_inicio, fecha_fin)
\end{verbatim}
Se conecta a Yahoo Finance descarga los históricos de dos tickers y devuelve la correlación de los log-rendimientos.

\item \begin{verbatim}
ultimo_precio(ticker, fecha_inicio, fecha_fin)
\end{verbatim}
Determina el último precio descargado para usarse como referencia de inicio para llevar a cabo las simulaciones.

\item \begin{verbatim}
arbol_americano(ticker, fecha_inicio, fecha_fin, K, T,r,N,tipo)
\end{verbatim}
Calcula el precio de una call/put tipo Americano mediante árbol binomial sobre un activo.
Requiere la tasa libre de riesgo r, el strike, el plazo, el número de subdivisiones del árbol y el tipo de de la opción

\item \begin{verbatim}
arbol_americano_portafolio(ticker1,
                            ticker2,
                        w, #Proporción del activo 1
                               fecha_inicio, fecha_fin, K, T,r,N,tipo)
\end{verbatim}
Calcula el precio de una call/put tipo Americano mediante árbol binomial sobre un portafolio de dos activos. Requiere la tasa libre de riesgo r, el strike, el plazo, el número de subdivisiones del árbol y el tipo de de la opción y la cantidad del activo 1 (w).

\item \begin{verbatim}
genera_simulaciones_individual(ticker1, fecha_inicio, fecha_fin, num_sims, dt, T)
\end{verbatim}
Genera las simulaciones. Requiere el número de simulaciones, el plazo y la longitud entre una observación y otra.

\item \begin{verbatim}
genera_df_simulaciones(ticker1, fecha_inicio, fecha_fin, num_sims, dt, T)
\end{verbatim}
Convierte las simulaciones en un data frame.

\item \begin{verbatim}
ejercicio_americana_LS(ticker1, 
fecha_inicio, 
fecha_fin, 
num_sims, dt, T, funcion_payoff, K, grados_pasado):
\end{verbatim}
Obtiene el momento de ejercito optimo de cada trayectoria usando el algoritmo de LS

\item \begin{verbatim}
calcula_precio_americana(ticker1, 
fecha_inicio, 
fecha_fin, 
num_sims, dt, T, 
funcion_payoff, K, grados_pasado):
\end{verbatim}
Calcula el precio de una call/put americana sobre un activo invodual usando el algoritmo de LS.

\item \begin{verbatim}
genera_simulaciones_correlacionadas(ticker1,ticker2,
fecha_inicio, fecha_fin,
num_sims, dt, T):
\end{verbatim}
Genera simulaciones de trayectorias de dos activos correlacionadas.

\item \begin{verbatim}
genera_df_portafolio(w, ticker1,ticker2, 
fecha_inicio, fecha_fin, num_sims, dt, T):
\end{verbatim}
Convierte la simulaciones del portafolio en un dataframe

\item \begin{verbatim}
ejercicio_americana_LS_port(w, ticker1,ticker2, 
fecha_inicio, fecha_fin, num_sims, dt, T, 
funcion_payoff, K, grados_pasado):
\end{verbatim}
Obtiene el momento de ejercito optimo de cada trayectoria usando el algoritmo de LS, requiere además la proporción a invertir en el activo 1.

\item \begin{verbatim}
calcula_precio_americana_port(w, ticker1,ticker2,
fecha_inicio, fecha_fin, num_sims, dt, T,
funcion_payoff, K, grados_pasado):
\end{verbatim}
Calcula el precio de una call/put americana sobre un portafolio de dos activos correlacionados usando el algoritmo de LS, requiere además la proporción a invertir en el activo 1.
\end{itemize}

\noindent \textbf{Resultados que no funcionalizados:}\\

\begin{itemize}
\item \begin{verbatim}
plt.hist(ejercicios_result, bins=int(T/2))
\end{verbatim}
Obtiene el histograma de los ejercicios óptimos, basados en LS, de cada trayectoria simulada.
\end{itemize}

\noindent \textbf{Funciones auxiliares:}\\

\begin{itemize}
\item \begin{verbatim}
payoff_call(x)
\end{verbatim}
Calcula $\max\{x-K,0\}$

\item \begin{verbatim}
payoff_call(x)
\end{verbatim}
Calcula $\max\{K-x,0\}$

\item \begin{verbatim}
vp_un(x)
\end{verbatim}
Calcula el valor presente a un periodo
\end{itemize}

\end{document}
